% TODO change verbatim into listings
% Clever thing to minimize the files I need to edit to change how the files are
% looking. The following one line comment is to trick VIM-LaTeX
% \documentclass{article}
%documentclass{scrartcl}
\KOMAoptions{
    fontsize=10pt
}
\setkomafont{pagenumber}{\bfseries\upshape\oldstylenums}
\renewcommand{\titlefont}{\rm\bfseries\LARGE}
\usepackage{
    ifxetex, 
    ifdraft,
    ifthen
}

\ifxetex
  \usepackage{fontspec}
  \usepackage{xunicode}
  \defaultfontfeatures{Mapping=tex-text} % To support LaTeX quoting style
  \setromanfont{Gentium}
\else
  \usepackage[utf8]{inputenc}
  \usepackage[T1]{fontenc}
  \usepackage{lmodern,textcomp}
\fi

\usepackage{
%    standalone,
%    lastpage,
    geometry,
    scrpage2,
%    setspace,
    amsmath,
    caption,
    calc,
%    floatrow,
}
\usepackage{
%    xcolor,
    graphicx,
    tikz,
    chemfig
}
\usepackage[final]{listings}
\usepackage[version=3]{mhchem}
\usepackage[update,verbose=false]{epstopdf}
\usepackage[colorinlistoftodos,obeyDraft]{todonotes}
\usepackage{hyperref}


% Set the geometry
\geometry{
    paper = a4paper,
    top=3cm,
    bottom=4cm,
    footskip=1cm,
    marginparwidth=3.5cm,
    headsep=1cm
}
\ifoptiondraft{
\geometry{inner=1.5cm, outer=4cm}
}{
\geometry{inner=3.0cm, outer=2.5cm}
}

%\onehalfspacing

% Setup hyperref
\hypersetup{
    colorlinks,
    urlcolor=blue,
    breaklinks
}

\usetikzlibrary{
    arrows,
    decorations.pathmorphing,
    backgrounds,
    positioning,
    fit,
    petri
}

% Define collors
\definecolor{myyellow}{HTML}{FFFAC9}
\definecolor{myyellowl}{HTML}{FFFBDD}

% Define lstlisting env
\lstset{
    language=[LaTeX]TeX,
    backgroundcolor=\color{myyellowl},
    numbers=left,
    numberstyle=\footnotesize,
    breaklines=true,
    breakatwhitespace=true,
    print=true
}

% Renew 2 styles
\renewpagestyle{plain}{{}{}{}}{{}{}
{\hfill\pagemark{}}}
\renewpagestyle{headings}{{}{}{}}{{}{}
{\hfill\pagemark{}}}

% Set the headings page style
\pagestyle{headings}

% Text mode commands
\newcommand{\uurl}[2]{\href{#1}{#2}\footnote{The URL is \url{#1}}}
\newcommand{\ftype}[1]{\texttt{.#1}}
\newcommand{\fname}[2]{\texttt{#1.#2}}
\newcommand{\pkg}[1]{\texttt{#1}}
\newcommand{\env}[1]{\texttt{#1}}
\newcommand{\cmd}[1]{\texttt{\textbackslash{}#1}}
\newcommand{\usepkg}[2]{
    \texttt{\textbackslash{}usepackage%
    \ifthenelse{\equal{#2}{}}{}{[#2]}\{#1\}}}
\newcommand{\comment}[1]{}

% New commands which ease the work. Units and relative uncertainties
\newcommand{\unit}[1]{\ensuremath{\, \mathrm{#1}}}
\newcommand{\rel}[1]{\ensuremath{ \cfrac{\Delta #1}{#1}}}
\newcommand{\eten}[1]{\ensuremath{ \times 10^{#1}}}
\newcommand{\DP}[2]{\ensuremath{\cfrac{\partial #1}{\partial #2}}}
\newcommand{\DD}[2]{\ensuremath{\cfrac{\mathrm{d} #1}{\mathrm{d} #2}}}
\newcommand{\dd}[1]{\ensuremath{\mathrm{d}#1}}

% Alter some LaTeX defaults for better treatment of figures:
    % See p.105 of "TeX Unbound" for suggested values.
    % See pp. 199-200 of Lamport's "LaTeX" book for details.
    %   General parameters, for ALL pages:
    \renewcommand{\topfraction}{0.9}    % max fraction of floats at top
    \renewcommand{\bottomfraction}{0.8} % max fraction of floats at bottom
    %   Parameters for TEXT pages (not float pages):
    \setcounter{topnumber}{2}
    \setcounter{bottomnumber}{2}
    \setcounter{totalnumber}{4}     % 2 may work better
    \setcounter{dbltopnumber}{2}    % for 2-column pages
    \renewcommand{\dbltopfraction}{0.9} % fit big float above 2-col. text
    \renewcommand{\textfraction}{0.07}  % allow minimal text w. figs
    %   Parameters for FLOAT pages (not text pages):
    \renewcommand{\floatpagefraction}{0.7}      % require fuller float pages
    % N.B.: floatpagefraction MUST be less than topfraction !!
    \renewcommand{\dblfloatpagefraction}{0.7}   % require fuller float pages

\captionsetup{
    format          = plain,        %
    labelformat     = simple,       %
    labelsep        = period,       %
    justification   = default,      %
    font            = default,      %
    labelfont       = {bf,sf},      %
    textfont        = default,      %
    margin          = 0pt,          %
    indention       = 0pt,          %
    parindent       = 0pt,          %
    hangindent      = 0pt,          %
    singlelinecheck = false         %
}

\renewcommand{\thefigure}{\oldstylenums{\arabic{figure}}}

\setatomsep{5mm}
\setbondoffset{.5mm}
\setcrambond{2.5pt}{1pt}{2pt}
\setbondstyle{thick}
\renewcommand*\printatom[1]{{\footnotesize\ensuremath{\mathsf{#1}}}}


% Custom stuff
\lstset{language=bash,numbers=none}
\newcommand{\MiKTeX}{MiK\TeX}
\newcommand{\MacTeX}{Mac\TeX}

% Title and similar stuff
\title{How to set up your computer to start using \LaTeX{}}
\author{Ignas Anikevicius}

\begin{document}

\maketitle

\section{Introduction}

Thank you for your interest in \LaTeX\ typesetting system and this article
will help you to get you ready for starting to use \LaTeX\ on your
computer.

Although I would like to write a continuous text on how to install
everything on different kinds of OSes (Operating Systems), I believe, that it
is not necessary to duplicate any content, if it can be found in a better
shape elsewhere. Therefore, I suggest you reading chapters of the book called
\LaTeX\ hosted on the website called wikibooks.org. You can find an on-line
version of the
\href{https://secure.wikimedia.org/wikibooks/en/wiki/LaTeX}{book} or the
\href{http://upload.wikimedia.org/wikipedia/commons/2/2d/LaTeX.pdf}{PDF}
version of it, which I think is much more suitable for reading or printing.

The list of the needed software is already there and if somebody feels very
comfortable with his system, no specific directions should be necessary for
them.

\section{Software from the Department of Chemistry}

Computer Office is already providing images for deploying the whole OS and
necessary software for Chemistry Department members. As far as I was informed,
there are images for Linux and Windows systems. For Macs, there might be
customized installers as well available on \url{http://www.google.co.uk}.

\section{\LaTeX\ distribution installation}

You need either of these:
\begin{itemize}
    \item ``\TeX\ Live'' \LaTeX\ distribution which is available for
        Linux/Mac/Windows, but should be preferred on Linux machines.
    \item ``\MacTeX'' \LaTeX\ distribution which is available for Mac machines
        only and should be the preferred option on these machines.
    \item ``\MiKTeX'' \LaTeX\ distribution which is available for Windows machines
        only and should be the preferred option on these machines.
\end{itemize}

\subsection{Notes for Linux users}

Use your Linux Distribution package manager whenever you can and install
``\TeX\ Live'' only from there. If you do not know how to do it, please refer to
your Distribution Wikipedia and search it for 'LaTeX' or 'TeX Live'.

Here is a list of most popular distributions and links to their Wikipedia Pages:

\begin{description}
    \item[.deb based] 
        \href{http://wiki.debian.org/Latex}{Debian},
        \footnote{The URL for Debian wiki is \url{http://wiki.debian.org/Latex}}
        \href{https://help.ubuntu.com/community/LaTeX}{Ubuntu},
        \footnote{The URL for Ubuntu wiki is
        \url{https://help.ubuntu.com/community/LaTeX}}
        and for distributions, which are derived from these two, the same wiki
        pages can be used. However, for full installation of \TeX\ Live you can try
        these terminal commands (issue them as root):
\begin{lstlisting}
aptitude update
aptitude install texlive-full
\end{lstlisting}

    \item[.rpm based]
        For Fedora, RedHat, CentOS and openSUSE distributions, use your package
        manager and install the full \TeX\ Live distribution. The following
        command executed in terminal as root should work:
\begin{lstlisting}
yum install texlive-full
\end{lstlisting}

    \item[ArchLinux and derivatives]
        The following instruction should work on ArchLinux and Chackra
        distributions. Both use pacman as their packages manager, so the
        following commands executed as root user will suffice:
\begin{lstlisting}
pacman -S texlive-most texlive-lang
\end{lstlisting}
        NB the second package texlive-lang is for the different languages
        support and if you use only English language, then you are free to
        install only the \verb|texlive-most| package

        Archlinux has a very good wiki article:
        \url{https://wiki.archlinux.org/index.php/TeX_Live}.
        
    \item[Gentoo and derivatives]
        This applies for Gentoo, Funtoo, Sabayon distributions. Things which
        work will definitely work on the other two distributions, so we will
        analyse only Gentoo.

        For checking the list of functionality for texlive distribution, enter:
\begin{lstlisting}
equery uses texlive
\end{lstlisting}
        and you will get the list of available use flags.
        You will have to enable the needed flags in the
        \verb|/etc/portage/package.use| and then just emerge the packages:
\begin{lstlisting}
emerge -av texlive
\end{lstlisting}
        In order to get a newer version of \TeX\ Live, just unmask the needed
        packages via \verb|/etc/portage/package.accept_keywords|.

        Gentoo has a very good
        \href{http://www.gentoo.org/proj/en/tex/texlive-migration-guide.xml}{Wiki}
        page documenting the installation.
    \item[Others]
        Install the \TeX\ Live distribution via your distributions packet
        manager. If you do not know how to do that, ask in the forums on your
        distribution web page.
\end{description}

\subsection{Notes for Mac users}

For easier experience, just install the full \MacTeX\ installation which can be
found on the following \href{http://www.tug.org/mactex/}{website}.
\footnote{The URL for the \MacTeX\ website is \url{http://www.tug.org/mactex/}}

\subsection{Notes for Windows users}

For easier experience, download \MiKTeX\ installation files from
\href{http://miktex.org/2.9/setup}{their website}.
\footnote{The URL for the website is \url{http://miktex.org/2.9/setup}}
There are mainly 2 wise options to select:
\begin{description}
    \item[Install everything] Although this might be very convenient as
        one will not have to worry about missing packages, but it takes space.
        On the other hand, slightly more than 1GB of occupied space on modern
        computers will not make a difference.
    \item[Install a base system] This is the alternative, which would take less
        space. What is more, one can select an option where necessary packages
        could be installed on the fly without any user intervention.
\end{description}

\section{Editing a .tex file}

Mainly there are two choices:
\begin{itemize}
    \item IDE (Integrated Development Environment)
        \footnote{\href{https://secure.wikimedia.org/wikipedia/en/wiki/Integrated_development_environment}{Wikipedia
        article}}
    \item Just a text editor.
\end{itemize}

While IDEs generally will provide a user with much more integrated environment,
this does not necessarily mean, that producing \LaTeX\ documents with an IDE is
generally faster. There are many very powerful text editors, which might have a
steep learning curve, but once mastered, they are very fast. What is more, some
text editors might be better in some tasks than other, so there is no such thing
as ``the best'' IDE or text editor for \LaTeX.

The most important projects are mentioned bellow:
\begin{description}
    \item[VIM \& Emacs] VIM is the best editor, in my opinion. It is very fast,
        lightweight and it can be customized a lot. Although it has a steep
        learning curve, it is very rewarding afterwards and reading any of the
        books on VIM would help a lot. 
        
        This being said, everybody admit, that Emacs is also good, and many
        argue that it is better than VIM. This has much to do with so-called
        editor wars.
        \footnote{Editor wars on
        \href{http://en.wikipedia.org/wiki/Editor_war}{Wikipedia}}

        Since both are very advanced editors, you will find that they have very
        powerful \LaTeX\ plug-ins, which might make the work faster than with
        most of the IDEs.

    \item[LyX] This is a project, that aims the user to give a word-processor,
        which would use \LaTeX\ internally. Although one can achieve really good
        results with it, technically you do not write \LaTeX\ and it will not
        help you at all with \LaTeX\ if you want to learn it. However, since it
        does a lot of automatic things, it might be a very good reference tool
        for searching hints how to achieve some things with \LaTeX\ (eg.
        searching for symbols, remembering commands). 
        
        That said, I have to insist on you that {\bfseries YOU DO NOT USE THIS
        WORD PROCESSOR, OTHER THAN FOR REFERENCE!} The reason is because
        publishers do not accept LyX files and once you export them to \LaTeX\,
        it becomes a mess. What is more, it will be easier to collaborate with
        colleagues if you use \LaTeX\. And sometimes, it tries to do more, than
        you want, or ask him to do and then you get errors, and spend so much
        precious time debugging instead of writing your thesis.

    \item[TeXShop] This is probably the first good IDE for Mac, which was highly
        successful and still is very popular.

    \item[TeXnicCenter] This is one of better IDEs for \LaTeX\ typesetting in
        Windows OS. 

    \item[TeXworks] A cross-platform IDE which was inspired by \textbf{TeXShop}.

    \item[TeXMaker] A good cross-platform IDE using Qt toolkit.
\end{description}

Other projects, which can be still very well used to achieve good results, but
are somewhat less popular:
\begin{description}
    \item[Kile] This is an IDE for Linux.
    \item[Geany] This is an IDE for Linux.
        %FIXME
    \item[Others] Need to add more.
\end{description}

\section{Bibliography management software}

Bibliography is usually managed through so-called Bib\TeX\ and there are various
GUIs (Graphical User Interfaces) to deal with such things. 

Some IDEs can interface with Bib\TeX, but may or may not require some additional
setup. As of how to do this, the best place to search would be the documentation
of your IDE of choice or some web-search engine (eg.
\href{http://www.google.co.uk}{Google}, \href{http://www.bing.co.uk}{Bing},
etc.)
Also some more advanced text editors can also do it (eg. VIM and Emacs).
However, there is a third category, which includes stand-alone software for
managing bibliographies (eg. Jabref, Bibdesk (OS X only), )
%FIXME need to find a list of tools for bib and Win

\section{PDF viewers}

Good PDF viewers are different across different platforms. I believe, that you
might want say, that Adobe's PDF viewer is very good, but the truth is that it
is slow and not as reliable as others.

A much better alternative might look \textbf{Foxit} PDF reader, which is
available for both Linux and Windows operating systems. However, by no means it
is the best solution and one should research a bit before settling down with the
most appealing PDF viewer.

\subsection{On Linux}

Linux users have a huge variety of PDF viewers to select from. One should search
distribution's repositories, but just to mention a few:
\begin{description}
    \item[Evince] Default for GNOME;
    \item[Okular] Default for KDE;
    \item[epdf]
    \item[zathura]
    \item[mupdf]
\end{description}

\subsection{On Mac}

The best choices seem to be viewers \textbf{Preview} and \textbf{Skim} as both
are relatively light and provide a good number of features. There might be
others, which I am not aware off as well.

\subsection{On Windows}

The best choice would be a \textbf{Sumatra} PDF viewer. Other alternatives
either need to be bought or they are not as reliable/complete as
\textbf{Sumatra} PDF viewer.

\section{Other useful software \& links}

\end{document}

% Editor configuration:
% vim: tw=80:spell:spelllang=en_gb
