% Clever thing to minimize the files I need to edit to change how the files are
% looking. The following one line comment is to trick VIM-LaTeX
% \documentclass{article}
%documentclass{scrartcl}
\KOMAoptions{
    fontsize=10pt
}
\setkomafont{pagenumber}{\bfseries\upshape\oldstylenums}
\renewcommand{\titlefont}{\rm\bfseries\LARGE}
\usepackage{
    ifxetex, 
    ifdraft,
    ifthen
}

\ifxetex
  \usepackage{fontspec}
  \usepackage{xunicode}
  \defaultfontfeatures{Mapping=tex-text} % To support LaTeX quoting style
  \setromanfont{Gentium}
\else
  \usepackage[utf8]{inputenc}
  \usepackage[T1]{fontenc}
  \usepackage{lmodern,textcomp}
\fi

\usepackage{
%    standalone,
%    lastpage,
    geometry,
    scrpage2,
%    setspace,
    amsmath,
    caption,
    calc,
%    floatrow,
}
\usepackage{
%    xcolor,
    graphicx,
    tikz,
    chemfig
}
\usepackage[final]{listings}
\usepackage[version=3]{mhchem}
\usepackage[update,verbose=false]{epstopdf}
\usepackage[colorinlistoftodos,obeyDraft]{todonotes}
\usepackage{hyperref}


% Set the geometry
\geometry{
    paper = a4paper,
    top=3cm,
    bottom=4cm,
    footskip=1cm,
    marginparwidth=3.5cm,
    headsep=1cm
}
\ifoptiondraft{
\geometry{inner=1.5cm, outer=4cm}
}{
\geometry{inner=3.0cm, outer=2.5cm}
}

%\onehalfspacing

% Setup hyperref
\hypersetup{
    colorlinks,
    urlcolor=blue,
    breaklinks
}

\usetikzlibrary{
    arrows,
    decorations.pathmorphing,
    backgrounds,
    positioning,
    fit,
    petri
}

% Define collors
\definecolor{myyellow}{HTML}{FFFAC9}
\definecolor{myyellowl}{HTML}{FFFBDD}

% Define lstlisting env
\lstset{
    language=[LaTeX]TeX,
    backgroundcolor=\color{myyellowl},
    numbers=left,
    numberstyle=\footnotesize,
    breaklines=true,
    breakatwhitespace=true,
    print=true
}

% Renew 2 styles
\renewpagestyle{plain}{{}{}{}}{{}{}
{\hfill\pagemark{}}}
\renewpagestyle{headings}{{}{}{}}{{}{}
{\hfill\pagemark{}}}

% Set the headings page style
\pagestyle{headings}

% Text mode commands
\newcommand{\uurl}[2]{\href{#1}{#2}\footnote{The URL is \url{#1}}}
\newcommand{\ftype}[1]{\texttt{.#1}}
\newcommand{\fname}[2]{\texttt{#1.#2}}
\newcommand{\pkg}[1]{\texttt{#1}}
\newcommand{\env}[1]{\texttt{#1}}
\newcommand{\cmd}[1]{\texttt{\textbackslash{}#1}}
\newcommand{\usepkg}[2]{
    \texttt{\textbackslash{}usepackage%
    \ifthenelse{\equal{#2}{}}{}{[#2]}\{#1\}}}
\newcommand{\comment}[1]{}

% New commands which ease the work. Units and relative uncertainties
\newcommand{\unit}[1]{\ensuremath{\, \mathrm{#1}}}
\newcommand{\rel}[1]{\ensuremath{ \cfrac{\Delta #1}{#1}}}
\newcommand{\eten}[1]{\ensuremath{ \times 10^{#1}}}
\newcommand{\DP}[2]{\ensuremath{\cfrac{\partial #1}{\partial #2}}}
\newcommand{\DD}[2]{\ensuremath{\cfrac{\mathrm{d} #1}{\mathrm{d} #2}}}
\newcommand{\dd}[1]{\ensuremath{\mathrm{d}#1}}

% Alter some LaTeX defaults for better treatment of figures:
    % See p.105 of "TeX Unbound" for suggested values.
    % See pp. 199-200 of Lamport's "LaTeX" book for details.
    %   General parameters, for ALL pages:
    \renewcommand{\topfraction}{0.9}    % max fraction of floats at top
    \renewcommand{\bottomfraction}{0.8} % max fraction of floats at bottom
    %   Parameters for TEXT pages (not float pages):
    \setcounter{topnumber}{2}
    \setcounter{bottomnumber}{2}
    \setcounter{totalnumber}{4}     % 2 may work better
    \setcounter{dbltopnumber}{2}    % for 2-column pages
    \renewcommand{\dbltopfraction}{0.9} % fit big float above 2-col. text
    \renewcommand{\textfraction}{0.07}  % allow minimal text w. figs
    %   Parameters for FLOAT pages (not text pages):
    \renewcommand{\floatpagefraction}{0.7}      % require fuller float pages
    % N.B.: floatpagefraction MUST be less than topfraction !!
    \renewcommand{\dblfloatpagefraction}{0.7}   % require fuller float pages

\captionsetup{
    format          = plain,        %
    labelformat     = simple,       %
    labelsep        = period,       %
    justification   = default,      %
    font            = default,      %
    labelfont       = {bf,sf},      %
    textfont        = default,      %
    margin          = 0pt,          %
    indention       = 0pt,          %
    parindent       = 0pt,          %
    hangindent      = 0pt,          %
    singlelinecheck = false         %
}

\renewcommand{\thefigure}{\oldstylenums{\arabic{figure}}}

\setatomsep{5mm}
\setbondoffset{.5mm}
\setcrambond{2.5pt}{1pt}{2pt}
\setbondstyle{thick}
\renewcommand*\printatom[1]{{\footnotesize\ensuremath{\mathsf{#1}}}}


% Custom packages

\title{Labels and Cross-referencing in \LaTeX{}}
\author{Ignas Anikevicius}

\begin{document}

\maketitle

This will be a short but \emph{very} meaningful tutorial on cross-references in
\LaTeX{}
%
What makes \LaTeX{} cross-referencing so good is that you can literally refer to
    almost any object you want, which makes it very useful.
%
The fact, that this task is trivial to do makes it very difficult to ignore it.

%    
Everything works via \verb|label| command, which defines a \emph{makrker}, to
    which you can refer later on via \verb|ref| command.
%
So for example, referencing to a section called ``Introduction'' you just have
    to specify it with the following code:
%
\begin{lstlisting}
\section{Introduction}
\label{sec:intro}
\end{lstlisting}
%
or
%
\begin{lstlisting}
\section{Introduction \label{sec:intro}}
\end{lstlisting}
%
Both code variants will work and when you would execute a command
    \verb|\ref{sec:intro}|, you would get a section number.
%
For getting a page-number you would similarly execute 
    \verb|\pageref{sec:intro}| command.

%
Now note the prefix I have given to the newly created label.
%
Since one can refer to almost any object, it might get slightly confusing once
    you have more than 10 references.
%
In order to avoid this, most \LaTeX{} users make labels with various prefixes,
    which denote what type of object it is.
%
Common prefixes are as follows:
%
\begin{itemize}
    \item \emph{sec:} for sections
    \item \emph{subsec:} for sections or you can use \emph{ssec:}
    \item \emph{eq:} for equations
    \item \emph{fig:} for figures
    \item \emph{tab:} for tables
\end{itemize}

Note, that the list of the prefixes is completely trivial, so you can use
    whatever prefixes you like.
%
If you use packages, which provide new commands, or environments, then it would
    be good idea to come up with prefixes for these things as well.
%
\begin{itemize}
    \item \emph{schem:} for schemes
    \item \emph{lst:} for code listings
    \item \emph{cheq:} for chemical reaction equations
\end{itemize}

\emph{NOTE}, the prefix convention is not obligatory, so one can not use it, but
    in the end, for bigger documents, it makes reading, and writing the document
    so much easier.
%
Therefore, it would be a good idea to use it \emph{always}.

For more information on cross-referencing, please read a chapter on
    \uurl{https://secure.wikimedia.org/wikibooks/en/wiki/LaTeX/Labels_and_Cross-referencing}{Labels
    and Cross-referencing} on the 
    \uurl{https://secure.wikimedia.org/wikibooks/en/wiki/LaTeX}{\LaTeX{} wikibook}.


\end{document}

% Editor configuration:
% vim: tw=80:spell:spelllang=en_gb


