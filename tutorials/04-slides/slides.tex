% Clever thing to minimize the files I need to edit to change how the files are
% looking. The following one line comment is to trick VIM-LaTeX
\documentclass[
    draft
]{scrartcl}
%documentclass{scrartcl}
\KOMAoptions{
    fontsize=10pt
}
\setkomafont{pagenumber}{\bfseries\upshape\oldstylenums}
\renewcommand{\titlefont}{\rm\bfseries\LARGE}
\usepackage{
    ifxetex, 
    ifdraft,
    ifthen
}

\ifxetex
  \usepackage{fontspec}
  \usepackage{xunicode}
  \defaultfontfeatures{Mapping=tex-text} % To support LaTeX quoting style
  \setromanfont{Gentium}
\else
  \usepackage[utf8]{inputenc}
  \usepackage[T1]{fontenc}
  \usepackage{lmodern,textcomp}
\fi

\usepackage{
%    standalone,
%    lastpage,
    geometry,
    scrpage2,
%    setspace,
    amsmath,
    caption,
    calc,
%    floatrow,
}
\usepackage{
%    xcolor,
    graphicx,
    tikz,
    chemfig
}
\usepackage[final]{listings}
\usepackage[version=3]{mhchem}
\usepackage[update,verbose=false]{epstopdf}
\usepackage[colorinlistoftodos,obeyDraft]{todonotes}
\usepackage{hyperref}


% Set the geometry
\geometry{
    paper = a4paper,
    top=3cm,
    bottom=4cm,
    footskip=1cm,
    marginparwidth=3.5cm,
    headsep=1cm
}
\ifoptiondraft{
\geometry{inner=1.5cm, outer=4cm}
}{
\geometry{inner=3.0cm, outer=2.5cm}
}

%\onehalfspacing

% Setup hyperref
\hypersetup{
    colorlinks,
    urlcolor=blue,
    breaklinks
}

\usetikzlibrary{
    arrows,
    decorations.pathmorphing,
    backgrounds,
    positioning,
    fit,
    petri
}

% Define collors
\definecolor{myyellow}{HTML}{FFFAC9}
\definecolor{myyellowl}{HTML}{FFFBDD}

% Define lstlisting env
\lstset{
    language=[LaTeX]TeX,
    backgroundcolor=\color{myyellowl},
    numbers=left,
    numberstyle=\footnotesize,
    breaklines=true,
    breakatwhitespace=true,
    print=true
}

% Renew 2 styles
\renewpagestyle{plain}{{}{}{}}{{}{}
{\hfill\pagemark{}}}
\renewpagestyle{headings}{{}{}{}}{{}{}
{\hfill\pagemark{}}}

% Set the headings page style
\pagestyle{headings}

% Text mode commands
\newcommand{\uurl}[2]{\href{#1}{#2}\footnote{The URL is \url{#1}}}
\newcommand{\ftype}[1]{\texttt{.#1}}
\newcommand{\fname}[2]{\texttt{#1.#2}}
\newcommand{\pkg}[1]{\texttt{#1}}
\newcommand{\env}[1]{\texttt{#1}}
\newcommand{\cmd}[1]{\texttt{\textbackslash{}#1}}
\newcommand{\usepkg}[2]{
    \texttt{\textbackslash{}usepackage%
    \ifthenelse{\equal{#2}{}}{}{[#2]}\{#1\}}}
\newcommand{\comment}[1]{}

% New commands which ease the work. Units and relative uncertainties
\newcommand{\unit}[1]{\ensuremath{\, \mathrm{#1}}}
\newcommand{\rel}[1]{\ensuremath{ \cfrac{\Delta #1}{#1}}}
\newcommand{\eten}[1]{\ensuremath{ \times 10^{#1}}}
\newcommand{\DP}[2]{\ensuremath{\cfrac{\partial #1}{\partial #2}}}
\newcommand{\DD}[2]{\ensuremath{\cfrac{\mathrm{d} #1}{\mathrm{d} #2}}}
\newcommand{\dd}[1]{\ensuremath{\mathrm{d}#1}}

% Alter some LaTeX defaults for better treatment of figures:
    % See p.105 of "TeX Unbound" for suggested values.
    % See pp. 199-200 of Lamport's "LaTeX" book for details.
    %   General parameters, for ALL pages:
    \renewcommand{\topfraction}{0.9}    % max fraction of floats at top
    \renewcommand{\bottomfraction}{0.8} % max fraction of floats at bottom
    %   Parameters for TEXT pages (not float pages):
    \setcounter{topnumber}{2}
    \setcounter{bottomnumber}{2}
    \setcounter{totalnumber}{4}     % 2 may work better
    \setcounter{dbltopnumber}{2}    % for 2-column pages
    \renewcommand{\dbltopfraction}{0.9} % fit big float above 2-col. text
    \renewcommand{\textfraction}{0.07}  % allow minimal text w. figs
    %   Parameters for FLOAT pages (not text pages):
    \renewcommand{\floatpagefraction}{0.7}      % require fuller float pages
    % N.B.: floatpagefraction MUST be less than topfraction !!
    \renewcommand{\dblfloatpagefraction}{0.7}   % require fuller float pages

\captionsetup{
    format          = plain,        %
    labelformat     = simple,       %
    labelsep        = period,       %
    justification   = default,      %
    font            = default,      %
    labelfont       = {bf,sf},      %
    textfont        = default,      %
    margin          = 0pt,          %
    indention       = 0pt,          %
    parindent       = 0pt,          %
    hangindent      = 0pt,          %
    singlelinecheck = false         %
}

\renewcommand{\thefigure}{\oldstylenums{\arabic{figure}}}

\setatomsep{5mm}
\setbondoffset{.5mm}
\setcrambond{2.5pt}{1pt}{2pt}
\setbondstyle{thick}
\renewcommand*\printatom[1]{{\footnotesize\ensuremath{\mathsf{#1}}}}


% Custom packages
\usepackage[]{cite}
\usepackage{framed}

\title{How to make presentations with \LaTeX{}}
\author{Ignas Anikevicius}

\begin{document}

\maketitle
\tableofcontents
\listoftodos
\vskip 1em

%
This tutorial is for the production of scientific presentations using the
    \LaTeX{} typesetting system.
%
There are numerous ways of making a presentation in \LaTeX{} mostly because
    there are a lot of views on how presentations should look like and how they
    should be prepared.
%
The most convenient way is to use common packages such as \pkg{beamer},
    \pkg{prosper}, \pkg{slides}, however, one can also tame the standard
    document classes, such as \pkg{scrartcl} \cite{PracTeX-komapres}.

%
There is also another source of information \cite{latexwikibook:presentations},
    which might be a good starting point for using the Beamer document class.
%
Also, only the \pkg{beamer} and the \pkg{scrartcl} ways of making presentations
    will be presented here as the first seems to be the most popular and
    reliable way of making presentations, whereas the second way brings more
    insights on how you can customize a class to typeset almost any document.

% ----------------------------------------------------------------------
\section{The beamer document class}
% ----------------------------------------------------------------------

%
As mentioned before, the \pkg{beamer} package is the most popular among
    scientists and it works quite well with PGF/TikZ packages which might make
    it the best solution out there.
%
It is very straight forward to use it --- it defines several new environments
    (e.g. \cmd{frame}, \cmd{columnt} etc.) and commands (e.g. \cmd{frametitle},
    \cmd{pause} etc.).

%
The best way to learn anything is by example.
%
Since this is not an exception, there are very good tutorials on the website
    which describe the use of \pkg{beamer} document class.
%
The list is as follows:
%
\begin{description}
    \item
        [\uurl{http://www.math.utah.edu/~smith/AmberSmith_GSAC_Beamer.pdf}{A
            Beamer tutorial by Amber Smith}] 
        %
        This is just a showcase of what \pkg{beamer} is capable of.
        %
        However, the author does not mention the \pkg{TikZ} package, which is
            actually written by the same author and it integrates with
            \pkg{beamer} very well.
    %
    \item
        [\uurl{http://www.uncg.edu/.../Charles\%20Batts\%20-\%20Beamer\%20Tutorial.pdf}{A
            Beamer tutorial by Charles Batts}] 
        %
        This tutorial explains the beamer capabilities very well and gives lots
            of examples.
        %
        I do think, that it is very suitable for learning and getting acquainted
            with \pkg{beamer}.
    %
    \item
        [\uurl{http://tug.org/pracjourn/2010-2/hofert.html}{Scientific
            Presentations using KOMA script classes}]
        %
        Although this link contains an article about making scientific
            presentation using a standard \pkg{scrartcl} class provided by KOMA
            script, it actually contains several very good references on how to
            make your presentations look nice and informative.
        %
        \pkg{Beamer} presentations might look much better with minimal amount of
            effort and this article becomes very useful as it points out some
            mistakes, which are made by inexperienced presenters.
        %
        Reading this will definitely help you to get professional presentations
            quickly.
    %
    \item
        [\uurl{http://www.tug.org/pracjourn/2005-4/mertz/}{Beamer by Example}]
        %
        This is yet another Prac\TeX{} article and it is also on teaching the
            reader how to use \pkg{beamer} class to get well looking
            presentations.
    %
    \item
        [\uurl{http://www.tug.org/pracjourn/2010-1/dohmen/dohmen.pdf}{Dual
            screen beamer presentations}]
        %
        This article give a very interesting insight on how one can adapt
            \pkg{beamer} for dual screen presentations.
        %
        Recommended for advanced \pkg{beamer} and \LaTeX{} users.
    %
\end{description}

%
By following the links to the Prac\TeX{} journal articles one could find
    example-files as well as sources for the article.
%
Examining those files might prove to be a very valuable experience and it is
    highly recommended to check out the examples if they are supplied in the
    archive file.

% ----------------------------------------------------------------------
\section{Using a Standard Document Class}
% ----------------------------------------------------------------------

%
This is an alternative method to create presentations.
%
It involves using \pkg{scrartcl} class which is then customized to a great
    extent.
%
The decorations of the presentation are done by using the \pkg{TikZ} package and
    since everything is done from scratch, the user will end up having a unique
    theme for his/her presentations.

%
As you might understand, with a lot of customizability comes slightly steeper
    learning curve and one need to spend more time initially to get everything
    set up.
%
However, then to convert an article to a presentation is much faster, which is
    mainly the point of this method.
%
Also, you do not need to learn knew commands/environments, which might also be
    considered as an advantage of this method.

%
There is an excellent example and article on the aforementioned
    PracTeX{} journal\cite{PracTeX-komapres}.
%
Please download the zipped sources where you will find everything.
%
You can check the presentation \ftype{tex} file and change various parameters in
    the preamble and see how the style of presentation is affected.

% ----------------------------------------------------------------------
\section{Thoughts on a Clever Work-flow}
% ----------------------------------------------------------------------

%
As might know from the mentioned resources, when making presentation one mostly
    has to divide the content to slides either by some sort of manual page
    breaks or by encapsulating everything in a \pkg{frame} environment which
    will automatically do that.
%
There is also another issue, that one might spend quite a lot of time while
    making a slide so that everything would look very good and redoing all the
    slides for every new presentation or talk one has to give might bee too time
    consuming.
%
These are the reasons why a clever working system is very important and it might
    decrease the time consumption substantially without sacrificing the quality
    of the slides.

%
\todo{Add a note about GROUP YOUR SLIDES}

%
Do not make all your slides in one file as it has several drawbacks:
\begin{description}
    \item[File becomes very large]
        It gets harder and harder to find things when one needs to edit
            something or do other kind of tinkering.
    %
    \item[Harder to reuse content]
        It is not very convenient to search in big files for information
            every time you need to access some information you have already
            typeset.
    %
    \item[Hard to deduce the length of a part]
        It is much harder to get an impression of the length of some section of
            the presentation as the file gets bigger because the line count of
            the file does not reflect the individual section sizes at all.
\end{description}

%
A much more clever approach would be to make slides in batches.
%
For example, if one has 4 sections in a presentation, then 4 different files
    will ease the organization substantially.
%
Some people even tend to have all slides as separate \ftype{tex} files, which
    get compiled when needed and then the resultant \ftype{pdf} files are used
    when needed.
%
This has a huge benefit, that every slide will look exactly as intended,
    however, it also means, that such features as table of contents or the
    highlighting of the current section must be either hard-coded or not shown
    at all.

%
Therefore, a good work-flow would consist of having different \ftype{tex} files
    for different subtopics and then combining them when needed.
%
The whole file and directory structure would be as follows:
\begin{lstlisting}
- archive
    | - Sn2mechanism.tex
    | - Sn2mechanism-HighE.tex
    | - Sn2mechanism-thermodynamics.tex
    |----------------------------------------------------------------
     This would include all the .tex files which contain different
     sections or even subsections which can be incorporated into one
     presentation very easily. One can even put single slides in
     different .tex files and the choice is mainly up to the user.

     The names of the .tex files should be unique and meaningful.

     If the user wants, he can split the folder into several smaller 
     for easier organization.
    |----------------------------------------------------------------
- 2002-04-12-SomeConference
    |- 2002-04-12-SomeConference.tex
  ......
- 2011-10-03-CambridgeFreshers
    | - 2011-10-03-CambridgeFreshers.tex
    |----------------------------------------------------------------
     All the presentations should be dated in order to find the
     easier and get them organized.
    
     Also files should have the same name. This way, the produced pdf
     file will have a meaningful name as well.
    |----------------------------------------------------------------
- figures_eps
- figures_ai
- figures_pdf
- figures_jpg
- figures_png
- figuers_...
    |----------------------------------------------------------------
     figure files should be put in different folders according to
     their extension. This helps to keep things tidy and easy to find.
     However, one should then ensure, that the figures are given
     meaningful and unique names.
    
     The figure paths, which would be used should be included via the
     /graphicspath{{dir1}{dir2}{dir3}} command
    |----------------------------------------------------------------
- references
    | Journals.bib
    | Books.bib
    | MyPublicationList.bib
    | ... etc ....
    |----------------------------------------------------------------
     Keep your bibliography databases here, so that it would be easy
     to find. Multiple files can be included at once if needed.
    |----------------------------------------------------------------
- macros.tex
    |----------------------------------------------------------------
     This file ideally should contain all the useful macros and
     package options so that the style of slides would be consistent.
     This file should be 
    |----------------------------------------------------------------
\end{lstlisting}
%
Using the \pkg{beamer} package would ensure that all text will appear on the
    slide and that it will be of required aspect-ratio and resolution.
%
What is more, it will give those very useful features like linking and table of
    contents inside the document.
%

    \todo{finish about this kind of work-flow}

% ----------------------------------------------------------------------
% \section{Bibliography}
\bibliographystyle{plain}
\bibliography{./../tutorial}
% ----------------------------------------------------------------------

\end{document}

% Editor configuration:
% vim: tw=80:spell:spelllang=en_gb
