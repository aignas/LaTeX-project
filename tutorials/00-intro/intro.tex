% Clever thing to minimize the files I need to edit to change how the files are
% looking. The following one line comment is to trick VIM-LaTeX
\documentclass[
    draft
    ]{scrartcl}
%documentclass{scrartcl}
\KOMAoptions{
    fontsize=10pt
}
\setkomafont{pagenumber}{\bfseries\upshape\oldstylenums}
\renewcommand{\titlefont}{\rm\bfseries\LARGE}
\usepackage{
    ifxetex, 
    ifdraft,
    ifthen
}

\ifxetex
  \usepackage{fontspec}
  \usepackage{xunicode}
  \defaultfontfeatures{Mapping=tex-text} % To support LaTeX quoting style
  \setromanfont{Gentium}
\else
  \usepackage[utf8]{inputenc}
  \usepackage[T1]{fontenc}
  \usepackage{lmodern,textcomp}
\fi

\usepackage{
%    standalone,
%    lastpage,
    geometry,
    scrpage2,
%    setspace,
    amsmath,
    caption,
    calc,
%    floatrow,
}
\usepackage{
%    xcolor,
    graphicx,
    tikz,
    chemfig
}
\usepackage[final]{listings}
\usepackage[version=3]{mhchem}
\usepackage[update,verbose=false]{epstopdf}
\usepackage[colorinlistoftodos,obeyDraft]{todonotes}
\usepackage{hyperref}


% Set the geometry
\geometry{
    paper = a4paper,
    top=3cm,
    bottom=4cm,
    footskip=1cm,
    marginparwidth=3.5cm,
    headsep=1cm
}
\ifoptiondraft{
\geometry{inner=1.5cm, outer=4cm}
}{
\geometry{inner=3.0cm, outer=2.5cm}
}

%\onehalfspacing

% Setup hyperref
\hypersetup{
    colorlinks,
    urlcolor=blue,
    breaklinks
}

\usetikzlibrary{
    arrows,
    decorations.pathmorphing,
    backgrounds,
    positioning,
    fit,
    petri
}

% Define collors
\definecolor{myyellow}{HTML}{FFFAC9}
\definecolor{myyellowl}{HTML}{FFFBDD}

% Define lstlisting env
\lstset{
    language=[LaTeX]TeX,
    backgroundcolor=\color{myyellowl},
    numbers=left,
    numberstyle=\footnotesize,
    breaklines=true,
    breakatwhitespace=true,
    print=true
}

% Renew 2 styles
\renewpagestyle{plain}{{}{}{}}{{}{}
{\hfill\pagemark{}}}
\renewpagestyle{headings}{{}{}{}}{{}{}
{\hfill\pagemark{}}}

% Set the headings page style
\pagestyle{headings}

% Text mode commands
\newcommand{\uurl}[2]{\href{#1}{#2}\footnote{The URL is \url{#1}}}
\newcommand{\ftype}[1]{\texttt{.#1}}
\newcommand{\fname}[2]{\texttt{#1.#2}}
\newcommand{\pkg}[1]{\texttt{#1}}
\newcommand{\env}[1]{\texttt{#1}}
\newcommand{\cmd}[1]{\texttt{\textbackslash{}#1}}
\newcommand{\usepkg}[2]{
    \texttt{\textbackslash{}usepackage%
    \ifthenelse{\equal{#2}{}}{}{[#2]}\{#1\}}}
\newcommand{\comment}[1]{}

% New commands which ease the work. Units and relative uncertainties
\newcommand{\unit}[1]{\ensuremath{\, \mathrm{#1}}}
\newcommand{\rel}[1]{\ensuremath{ \cfrac{\Delta #1}{#1}}}
\newcommand{\eten}[1]{\ensuremath{ \times 10^{#1}}}
\newcommand{\DP}[2]{\ensuremath{\cfrac{\partial #1}{\partial #2}}}
\newcommand{\DD}[2]{\ensuremath{\cfrac{\mathrm{d} #1}{\mathrm{d} #2}}}
\newcommand{\dd}[1]{\ensuremath{\mathrm{d}#1}}

% Alter some LaTeX defaults for better treatment of figures:
    % See p.105 of "TeX Unbound" for suggested values.
    % See pp. 199-200 of Lamport's "LaTeX" book for details.
    %   General parameters, for ALL pages:
    \renewcommand{\topfraction}{0.9}    % max fraction of floats at top
    \renewcommand{\bottomfraction}{0.8} % max fraction of floats at bottom
    %   Parameters for TEXT pages (not float pages):
    \setcounter{topnumber}{2}
    \setcounter{bottomnumber}{2}
    \setcounter{totalnumber}{4}     % 2 may work better
    \setcounter{dbltopnumber}{2}    % for 2-column pages
    \renewcommand{\dbltopfraction}{0.9} % fit big float above 2-col. text
    \renewcommand{\textfraction}{0.07}  % allow minimal text w. figs
    %   Parameters for FLOAT pages (not text pages):
    \renewcommand{\floatpagefraction}{0.7}      % require fuller float pages
    % N.B.: floatpagefraction MUST be less than topfraction !!
    \renewcommand{\dblfloatpagefraction}{0.7}   % require fuller float pages

\captionsetup{
    format          = plain,        %
    labelformat     = simple,       %
    labelsep        = period,       %
    justification   = default,      %
    font            = default,      %
    labelfont       = {bf,sf},      %
    textfont        = default,      %
    margin          = 0pt,          %
    indention       = 0pt,          %
    parindent       = 0pt,          %
    hangindent      = 0pt,          %
    singlelinecheck = false         %
}

\renewcommand{\thefigure}{\oldstylenums{\arabic{figure}}}

\setatomsep{5mm}
\setbondoffset{.5mm}
\setcrambond{2.5pt}{1pt}{2pt}
\setbondstyle{thick}
\renewcommand*\printatom[1]{{\footnotesize\ensuremath{\mathsf{#1}}}}


% Customize
\renewcommand{\quote}[1]{
\begin{center}
    \colorbox{myyellowl}{
    \begin{minipage}[t]{.92\textwidth}
        #1
    \end{minipage}
    }
\end{center}
}

\title{\LaTeX{} Fundamentals}
\author{Ignas Anikevičius}

\begin{document}

\maketitle

\tableofcontents
\listoftodos{\vskip 1em}

%
In this document you can find information on the key features of the \LaTeX{}
    typesetting system and some thoughts on the philosophy behind them.
%
If this is the first time you hear about \LaTeX{}, then it is quite important
    for you to feel comfortable with the basic structure and ideas of this
    typesetting system.
%
It might be also very useful to know some history behind \LaTeX{} as it might
    help you acknowledge the differences between \TeX{} and \LaTeX{}
%
If you feel, that you need more information, please refer to the \LaTeX{}
    wikibook, which is one of the best on-line resources for \LaTeX{}, or Leslie
    Lamport's book on \LaTeXe{}.
%
The link to this book can be found either on the Chemistry Department's \LaTeX{}
    website, or at the end of document.
    \todo{Create a Proper Citation for Leslie Lamport?}

% ----------------------------------------------------------------------
\section{Some History}
% ----------------------------------------------------------------------

%
The introduction to the Wikipedia article about \TeX{} typesetting system
    describes the \TeX{} system very well in just two sentences:
%
\quote{
\TeX{} is a typesetting system designed and mostly written by Donald Knuth.
    \vskip .5em

Together with the METAFONT language for font description and the Computer Modern
    family of typefaces, TeX was designed with two main goals in mind: to allow
    anybody to produce high-quality books using a reasonable amount of effort,
    and to provide a system that would give exactly the same results on all
    computers, now and in the future.
    }

%
\TeX was a very powerful language, but it did not have any powerful macros
    bundled with it, so people had to write macros, which was not very exciting
    thing to do.
%
In this way \LaTeX{} was created, which is just a set of macros written by
    Leslie Lamport for his own project and released as a package, so that other
    people could reuse them.
%
This short paragraph by Leslie Lamport himself summarizes the reasons behind its
    creation very well on his
    \uurl{http://research.microsoft.com/en-us/um/people/lamport/pubs/pubs.html\#latex}{website}:

% Input the file containing the quote. Very useful to keep things tidy\ldots
\quote{In the early 80s, I was planning to write the Great American Concurrency
Book.  
%
I was a TeX user, so I would need a set of macros.  
%
I thought that, with a little extra effort, I could make my macros usable by
others.  
%
Don Knuth had begun issuing early releases of the current version of TeX,
and I figured I could write what would become its standard macro package.  
%
That was the beginning of LaTeX.  
%
I was planning to write a user manual, but it never occurred to me that
anyone would actually pay money for it.  
%
In 1983, Peter Gordon, an Addison-Wesley editor, and his colleagues visited
me at SRI.  
%
Here is his account of what happened. 

\quote{
    \em
    Our primary mission was to gather information for Addison-Wesley "to
    publish a computer-based document processing system specifically
    designed for scientists and engineers, in both academic and professional
    environments." 
    %
    This system was to be part of a series of related products (software,
    manuals, books) and services (database, production).  
    %
    (La)TeX was a candidate to be at the core of that system.  (I am quoting
    from the original business plan.)  
    %
    Fortunately, I did not listen to your doubt that anyone would buy the
    LaTeX manual, because more than a few hundred thousand people actually
    did.  
    %
    The exact number, of course, cannot accurately be determined, inasmuch
    as many people (not all friends and relatives) bought the book more than
    once, so heavily was it used.  
    }

Meanwhile, I still haven't written the Great American Concurrency Book.  
}

% ----------------------------------------------------------------------
\section{Concept of Declarative Formatting}
% ----------------------------------------------------------------------

%
The reason why \LaTeX{} was created is just to further emphasize the point made
    by \TeX{}:
%
\quote{the content and the formatting should be as separate as possible.}

%
The above statement implies the following ideas:
\begin{itemize}
    \item The author can focus on the text more and he does not get distracted
        by temptation to change the looks of the contents by pressing various
        buttons and tweaking so that it appears nice in the text \emph{before}
        final version is ready.

    \item One should make the formatting of the document meaningful. What I want
        to say, is that for example to emphasize some word you do not put a
        command 
\begin{lstlisting}
\textit{emphasis}
\end{lstlisting}
        , which would look like
        \emph{emphasis}, but you'd rather use a existing command (or
        create/redefine one), so that you would write something like
\begin{lstlisting}
\emph{emphasis}
\end{lstlisting}
        or 
\begin{lstlisting}
\emphas{emphasis}
\end{lstlisting}
        and you get the same.

        Well this has several advantages:
        \begin{enumerate}
            \item If you need to change the looks of how you emphasize the words
                in the text, you just change the definition of the command you
                are already using.
                %
                This simple change might save you a lot in the long run.

            \item You do not have to think about how it will look in text if you
                want to put emphasis on the text.
                %
                You just know, that that word will be emphasized and that's it.

            \item The built in command \verb|\emph| already takes care of the
                fact that if you want to emphasize some text in already
                emphasized text, it will still get emphasized.
                
                For example:
                
                \emph{This is \emph{very} important text}.
                
                can be achieved by:
\begin{lstlisting}
\emph{This is \emph{very} important text}.
\end{lstlisting}
        \end{enumerate}
    \item Since the formatting is completely separated from the content, it is
        very easy to make the document look nice if the author has followed the
        \LaTeX{} conventions and the formatting of the text is abstract.

\end{itemize}

There is a very great short article on this if you follow
    \uurl{http://web.science.mq.edu.au/~rdale/resources/writingnotes/declform.html}{this
    link}.

% ----------------------------------------------------------------------
\section{Reading List Before Starting Your First Document}
% ----------------------------------------------------------------------

%
You can find a very good
    \uurl{https://secure.wikimedia.org/wikibooks/en/wiki/LaTeX}{on-line book}
    which can help you get started. 
%
Before continuing any further, you should read or at least skim through the
    first 5 chapters:
%
\begin{description}
    \item[Introduction] Mainly what is \LaTeX{} and what software is needed for
        it. It may serve as an addition to what is already written in the
        'setup' tutorial;
    \item[Absolute Beginners] Definitely read this if you hear the term \LaTeX{} the first
        time, although others should also skim through this as it might
        consolidate your knowledge;
    \item[Basics] The same as above chapter;
    \item[Document Structure] This might be more useful for \emph{Absolute
        Beginners};
    \item[Errors and Warnings] A very important chapter which might help you a
        lot to understand what your \LaTeX{} program is telling you when you
        make a mistake. It is advised to read it if you have problems with some
        \LaTeX{} documents before reporting to anyone;
\end{description}
%
If you like to have real books, check your preferred library for the newest
    version of ``\LaTeX{} - User's Guide and Reference Manual'' by Leslie
    Lamport.
%
It should have a note, that it is updated for the \LaTeXe{}, which is just the
    most recent version of \LaTeX{}.
%
And if you are thinking of buying one, consider buying this book first as
    nothing can be better than the \LaTeX{} author himself explaining how to use
    the system.

%
After having reading all the information I have presented here, I hope, that you
    will be able to appreciate how \LaTeX{} is done and it will be easier for you
    to start using \LaTeX{} the right way and after some time to notice, that
    producing document in \LaTeX{} takes you less time than in any other tool.

% ----------------------------------------------------------------------
\section{Description of the Tutorials in this Repository}
% ----------------------------------------------------------------------

\todo{Put actual links to the tutorials?}

%
Here I will outline general contents of the repository so that it would be
    easier to find various topics which are described in numerous tutorials I
    produced over the summer.

%
\subsection{How to Setup Your Machine}

%
In this tutorial I have compiled a lot of information on how to prepare you
    machine to use \LaTeX{}.
%
In there you will find everything you need to know about a \LaTeX{}
    distribution, editing with an IDE or a simple text editor, managing your
    bibliography database using sophisticated tools.

%
\subsection{Intermediate Skills}

%
There are number of tutorials which where written with chemists in mind, so
    there are tips on how to deal with problems, which might arise for the
    majority of you writing reports, papers and other documents.

%
\subsubsection{Managing Abbreviations and Glossaries Cleverly}

%
If you have ever tried to put together a big document and you needed to ensure
    the integrity of all abbreviations you used and then you needed to compile a
    list of it, then you will appreciate the usefulness of this tutorial.
%
It deals with automated, cross-referenced usage of abbreviations and, moreover,
    it gives you directions how to manage glossaries with some specific \LaTeX{}
    packages.

%
\subsubsection{Managing Bibliography}

%
In this tutorial I have described how to use different packages for bibliography
    management and citations and it encapsulates most of the needed commands for
    your everyday typesetting needs.
%
If you do not know how to make the reference list to appear in the way you want,
    or you want something like footnoted citations, then you will find some
    directions or code in this tutorial.
    
%
\subsubsection{Cross-referencing in \LaTeX{}}

%
\LaTeX{} is very useful when it comes to cross referencing.
%
However, its vast capabilities might sometimes make you feel clumsy or lost, so
    it is quite important to get some of the concepts right from the first time.
%
This tutorial is only 2 pages long, so it should not take too much time to skim
    through.

%
\subsubsection{Mathematical and Chemical Equations}

%
In this tutorial I have made some notes on how to produce different formats of
    equations and how to cross-reference them.
%
If you haven't used \pkg{mhchem} package, then you should check out the
    subsection on quick typesetting of the equations of chemical reactions.
    
%
\subsubsection{Different Fonts and Encodings in \LaTeX{}}

%
In this tutorial I am talking on how to customize your \LaTeX{} document in
    order to get required fonts.
%
There is also section on how to incorporate Unicode characters into text if
    needed (although \LaTeX{} has very powerful macros to typeset almost any
    character you might want).
%
There you can find information on alternative \LaTeX{} compilers, which for more
    technically inclined might be something worth trying out.

%
\subsubsection{Including Graphics, Producing Graphics in \LaTeX{}}

%
This is probably one of the most useful tutorials for scientists dealing with a
    lot of graphics.
%
It covers basics of how to include simple figures and it also includes some
    notes on how to overlay any figures with latex code.
%
The overlaying method is done with a very powerful \pkg{tikz} package, which can
    be even used to create entire figures if needed.
    
%
\subsubsection{Table Creation}

%
In this tutorial I have given some quick notes and examples on how to make
    appealing tables using several \LaTeX{} packages, such as \pkg{booktabs} and
    \pkg{array}.
%
It also includes information on how to include tables, which span over both
    columns in two-column documents.

%
\subsection{Template-Related Things}

%
\subsubsection{Making Posters}

%
This tutorial quickly reviews some possible solutions for poster production in
    \LaTeX{}.
%
Since there are numerous ways how to make a poster, there is not a single
    perfect solution which might suit all, hence this is more or less a
    compilation of notes where to find information and what packages to use.

%    
\subsubsection{Producing Presentations}

%
Presentations, like posters, are very personal, hence in this tutorial there are
    some notes on how to use several packages and there is some discussion on
    the work-flow on making the slides.
%
However, since it is very narrow field, it would be really hard to give you one
    big tutorial on how to achieve things and, therefore, you will find more or
    less guidelines and links to useful resources on presentation making in
    \LaTeX{}.

%
\subsubsection{Template Instructions}

%
In this tutorial you can find some notes on how to use the templates found on
    this website.
%
I hope that it will complement to the documentation already existing in the
    templates quite well and you will be able to use the templates without any
    troubles.

%
\subsection{Advanced Topics}

%
There are two more tutorials, which would be more suitable for people who
    already know everything from the tutorials listed above and would like to
    start creating their own macros or find out about tools which can make your
    work easier.
    
%
\subsubsection{Customizing \LaTeX{} to Your Own Needs: Defining Macros}

%
This tutorial is on defining your own macros which would make your text more
    meaningful.
%
For example if you want to emphasize a package name by using a bold-face font,
    then you should create your own macro for that with a name \verb|pkg| or
    something similar.
%
This helps in two ways: first, you do not have to worry about making the looks
    of all package names consistent, and second, you give to your text more
    meaning as it can be clearly seen that what you emphasize is a package name.
%
Second part of the very same tutorial is about counters and using them to
    enumerate chemical structures combining with the packages mentioned in the
    graphics tutorial.

%
\subsubsection{Getting Incremental Updates and other Goodies}

%
This tutorial is on Version Control System usage with \LaTeX{} which in
    simple-world terms means easier collaborative work and access to almost any
    version of the file at any time.
%
This tutorial was written in order to make git version control system more
    accessible to chemists and to provide an alternative (most say, that it is a
    better one) to the old Subversion (SVN).
%
However, the Computing Officers have already created a SVN server in the
    Department, so using it might be your only choice as of now.
%
For those who want invest time into newer and more advanced technology (git)
    there are several notes on how to make that happen in the nearest future.

% ----------------------------------------------------------------------
\section{List of Useful Links and Resources}
% ----------------------------------------------------------------------

%
Here is a list of very useful internet resources on \LaTeX{} typesetting system:
\begin{itemize}
    \item
        \uurl{http://www.eng.cam.ac.uk/help/tpl/textprocessing/}{CUED \LaTeX\
        website} very good resource for some \LaTeX\ related matters;
    \item
        \uurl{http://tug.org/pracjourn/2010-2/toc.html}{Prac\TeX{} journal} is a
        very good place for novel uses of \LaTeX{} and one can find very
        interesting tutorials there;
    \item
        \uurl{http://stackoverflow.com/questions/tagged/latex}{stack\textbf{overflow}
        archive} of answers to various \LaTeX{} related questions.
    \item
        \uurl{http://stackoverflow.com/questions/193298/best-practices-in-latex/196724}{Good
        \LaTeX{} practices} on stack\textbf{overflow}.
    \item
        The aforementioned 
        \uurl{https://secure.wikimedia.org/wikibooks/en/wiki/LaTeX}{\LaTeX{}
        wikibook}
    \item
        \uurl{http://www.ctan.org/}{CTAN \LaTeX{} repository}. This is the major
        database of packages and document classes. If you do not find what you
        want here, probably it does not exist (yet).
    \item
        \uurl{http://www.tug.dk/FontCatalogue/}{The \LaTeX{} font catalogue}. A
        very good place to know how to get fonts working.
    \item
        \uurl{http://www.texample.net/}{\TeX{}amples page}. A good resource on
        TikZ graphics package usage. It contains a lot of examples and might be
        the best way to start learning it just by examining everything.
    \item
        \uurl{http://www.tug.org/metapost.html}{Metapost related links}. This is
        yet another way to produce good quality scalable graphics. This library
        is based on the MetaFont library by Donald Knuth (the \TeX{} father).
\end{itemize}


\end{document}

% Editor configuration:
% vim: tw=80:spell:spelllang=en_gb
