\chapter*{Summary}
One possibility for controlling the high density organisation of enzymes for technological purposes, which could increase human use of sunlight and absorb atmospheric CO\sub{2}, would be to decorate amyloid fibres with functional enzymes. Therefore several multi-domain fusion proteins were made which consisted of permutations of the SH3 fibre forming domain and cytochrome b\sub{562}---an electron transfer domain whose native state is stabilised when it binds the electron transfer cofactor ferrous proto-porphyrin IX (also known as haem). It was anticipated that these fusion proteins would form fibres displaying the electron transfer enzymes and would be capable of transporting charge along the complete length of the fibre.  To understand the organisation of the proteins in a fibre, at a molecular level, a variety of experimental and theoretical approaches were required.

Most permutations of the fusion proteins successfully formed amyloid fibres, whose ribbon-like morphology varied on a transition between twisted ribbons and spiral ribbons, with notable exceptions. It is known that the helical fibres are formed from helical filaments consisting of beta strands that are perpendicular to the fibre axis. A simple model was created based on maximising the projection of adjacent beta strands, modelled as short rigid rods, on a helical space curve. Such maximisation occurs when the beta strands are perpendicular to the helical space curve and parallel to an arbitrary global reference plane, a condition satisfied by the normal vector of the Frenet frame for a helix. Filaments formed in this way can be coaxially aligned at varying displacements to recreate both twisted and spiral ribbon morphologies.

The variation in morphology, driven by the variation in sequence of the fusion proteins, was quantified by low resolution TEM and could be explained in the context of the model by distortion of the position of the beta strands due to excess material which was not involved in the core of the fibre. Low resolution AFM was used to confirm these trends and discovered that surface adsorption flattens the fibres against the substrate such that filament crossover points form features which enumerate the number of filaments in the fibre. Adsorption can also disintegrate the fibres into segments of characteristic length due to strain on the filaments at the crossover points.

High resolution AFM under liquid found that the surface roughness of the fibre depended on the species under consideration, the quantity of haem bound by the cytochromes and also confirmed the dimensions and number of filaments within each fibre species.

A key observation from preliminary work---that only half of the fibre-borne cytochromes can bind haem---was confirmed and extended to all the fibre forming proteins studied. The haem binding improves the imaging properties of the fibre and changes the morphology in a systematic way, thus yielding small molecule control over the fibre morphology which, in principle, could be used to bring the displayed cytochromes into adequate proximity for electron transfer to occur.

The improved imaging arising from haem binding allowed the high resolution AFM tip to penetrate deeper into the fibre. The existence of an inner filament linked to surface moeities was confirmed and the best information possible was used to create a model of the fibre.

The fibres were probed with STM and found to be insulating regardless of the presence of haem. Some conducting features were observed but could not be identified as fibres unequivocally. The formation of an ionic aqueous film combined with the rapid cycling of fusion proteins between the fibres and the solution inhibited the STM analysis of the fibres. 

